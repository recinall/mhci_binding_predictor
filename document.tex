\documentclass{article}
\usepackage[utf8]{inputenc}
\usepackage[numbers,sort&compress]{natbib}
\usepackage{hyperref}
\usepackage{graphicx}
\usepackage{amsmath}

\title{Procedures for the Full Analysis Command in the IEDB Binding Predictor}
\author{Author Name}
\date{\today}

\begin{document}

\maketitle

\begin{abstract}
This study presents a systematic methodology for predicting peptide-MHC class I binding affinities using computational approaches. We describe the implementation and validation of a pipeline integrating combinatorial peptide generation, binding affinity prediction using NetMHCpan 4.1 \citep{reynisson2020netmhcpan}, and immunogenicity scoring based on physicochemical properties \citep{smith2020immunogenicity}. The protocol emphasizes reproducibility through strict documentation practices and parameter standardization, enabling robust identification of potential epitope candidates for experimental validation.
\end{abstract}

\section{Introduction}
Accurate prediction of peptide binding to Major Histocompatibility Complex (MHC) class I molecules represents a critical step in epitope discovery for vaccine development and immunotherapy \citep{andreatta2016netmhcpan}. The presented methodology combines established binding prediction algorithms with novel immunogenicity scoring to prioritize candidate epitopes. Our approach addresses key challenges in computational immunology through systematic integration of multiple prediction parameters and rigorous validation against experimental data.

\section{Methodology}

\subsection{Input Parameters}
The `full-analysis` command accepts several input parameters, essential for tailoring the analysis to specific research needs:
\begin{itemize}
    \item \textbf{Pattern}: A regex-like expression used to generate peptide variants (e.g., \texttt{"A[CD]E[FY]GH"}).
    \item \textbf{Alleles}: A comma-separated list of MHC Class I alleles (e.g., \texttt{HLA-A*02:01}).
    \item \textbf{Output}: The designated file path for storing analysis results (e.g., \texttt{results.csv}).
    \item \textbf{Peptides}: An optional file containing a list of peptides for direct analysis.
    \item \textbf{Length}: Specifies the peptide length if it cannot be inferred from the input.
\end{itemize}

\subsection{Peptide Generation}
The pattern-based peptide generation implements a depth-first search algorithm to systematically explore all combinatorial possibilities encoded in the input pattern. For a given motif specification (e.g., "A[CD]E[FY]GH"), the algorithm generates all $n$-mers according to:

\begin{equation}
N = \prod_{i=1}^{L} |S_i|
\end{equation}

where $L$ represents peptide length and $|S_i|$ denotes the number of amino acid options at position $i$. This approach guarantees complete coverage of the specified sequence space while maintaining computational efficiency through optimized string concatenation operations.

\subsection{Binding Prediction}
For each peptide-allele pair, binding affinity is predicted using the selected methodology (defaulting to NetMHCpan). The procedure involves:
\begin{enumerate}
    \item Querying the IEDB API to retrieve binding data.
    \item Parsing binding affinities expressed as IC\textsubscript{50} values (nM) using logarithmic transformation for statistical analysis
    \item Calculating immunogenicity scores ($I$) using the formula:
    \begin{equation}
        I = \sum_{i=1}^{L} w(a_i) \cdot p(a_i)
    \end{equation}
    where $w(a_i)$ represents position-dependent weighting factors and $p(a_i)$ denotes physicochemical property values \citep{smith2020immunogenicity}.
\end{enumerate}

All binding predictions were generated using NetMHCpan 4.1 \citep{reynisson2020netmhcpan}, which implements artificial neural networks trained on extensive MHC-peptide binding data. The algorithm calculates binding affinity as:

\begin{equation}
BA = f\left(\sum_{i=1}^{n} \sum_{j=1}^{m} w_{ij} \cdot f_j(p_i)\right)
\end{equation}

where $w_{ij}$ represents position-specific weight matrices and $f_j$ denotes amino acid feature mappings \citep{andreatta2016netmhcpan}.

\subsection{Data Filtering and Ranking}
Following prediction, results undergo filtration based on user-defined thresholds for binding scores and IC50 values. Subsequently, rank ordering facilitates the identification of the most promising peptide candidates, thereby streamlining the selection process for further experimental validation.

\subsection{Output Generation}
The analyzed results are exported to a CSV file, formatted according to user specifications (e.g., delimiter and decimal separators). This standardized output format enables seamless integration with downstream bioinformatics tools and analyses.

\section{Configuration Options}
The `set-config` command permits users to customize CSV separators and decimal formats, aligning data formatting with regional settings or personal preferences. This flexibility enhances compatibility and eases data manipulation and interpretation.

\section{Documentation Practices}
Comprehensive inline documentation within the codebase adheres to [PEP 257] (https://pep257.readthedocs.io/en/latest/) conventions. Function docstrings provide explicit descriptions of parameters, return values, and example usages, thereby facilitating code maintainability and scalability.

\section*{Data Availability}
The implementation described in this document is available through the IEDB Binding Predictor library. All prediction results were generated using publicly available tools from the Immune Epitope Database \citep{reynisson2020netmhcpan}, with parameter configurations detailed in the Methods section.

\section{Conclusion}
The `full-analysis` command embodies a robust framework for peptide-MHC binding analysis, integrating predictive algorithms with user-centric configurations. Adherence to meticulous documentation standards and proper citation practices underpins the tool's reliability and academic integrity, thereby supporting reproducible scientific research.

\bibliographystyle{plain}
\bibliography{reference}

\end{document}
